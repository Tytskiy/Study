\documentclass[12pt,fleqn]{article}

\usepackage[utf8]{inputenc}
\usepackage[T2A]{fontenc}
\usepackage{amssymb,amsmath,mathrsfs,amsthm}
\usepackage[russian]{babel}
\usepackage[pdftex]{graphicx}
\usepackage{multirow}
\usepackage{indentfirst}
\usepackage[colorlinks,linkcolor=blue(ryb),citecolor=blue(ryb), unicode]{hyperref}

\usepackage{xcolor}
\usepackage{sectsty}

\usepackage{amsmath}
\usepackage{systeme}
\definecolor{blue(ryb)}{rgb}{0.2, 0.2, 0.6}

\usepackage{listings}
\usepackage{xcolor}

\definecolor{codegreen}{rgb}{0,0.6,0}
\definecolor{codegray}{rgb}{0.5,0.5,0.5}
\definecolor{codepurple}{rgb}{0.58,0,0.82}
\definecolor{backcolour}{rgb}{0.95,0.95,0.92}

\lstdefinestyle{mystyle}{
    backgroundcolor=\color{backcolour},   
    commentstyle=\color{codegreen},
    keywordstyle=\color{magenta},
    numberstyle=\tiny\color{codegray},
    stringstyle=\color{codepurple},
    basicstyle=\ttfamily\footnotesize,
    breakatwhitespace=false,         
    breaklines=true,                 
    captionpos=b,                    
    keepspaces=true,                 
    numbers=left,                    
    numbersep=5pt,                  
    showspaces=false,                
    showstringspaces=false,
    showtabs=false,                  
    tabsize=2
}

\lstset{style=mystyle}


%\usepackage[ruled,section]{algorithm}
%\usepackage[noend]{algorithmic}
%\usepackage[all]{xy}

% Параметры страницы
\sectionfont{\color{blue(ryb)}}
\subsubsectionfont{\color{blue(ryb)}}
\subsectionfont{\color{blue(ryb)}}
\textheight=24cm % высота текста
\textwidth=16cm % ширина текста
\oddsidemargin=0pt % отступ от левого края
\topmargin=-2.5cm % отступ от верхнего края
\parindent=24pt % абзацный отступ
\parskip=0pt % интервал между абзацам
\tolerance=2000 % терпимость к "жидким" строкам
\flushbottom % выравнивание высоты страниц
%\def\baselinestretch{1.5}
\setcounter{secnumdepth}{0}
\renewcommand{\baselinestretch}{1.1}

\newcommand{\norm}{\mathop{\rm norm}\limits}
\newcommand{\real}{\mathbb{R}}

\newcommand{\ex}{\mathbb{E}}
\newcommand{\diag}{\mathrm{diag}}
\newcommand{\intset}{\mathrm{int}}
\newcommand{\softmax}{\mathop{\rm softmax}\limits}
\newcommand{\lossfunc}{\mathcal{L}'}
\newcommand{\elbo}{\mathcal{L}}
\newcommand{\normal}[3]{\mathcal{N}(#1 | #2, #3)}
\newcommand{\dd}[2]{\frac{\partial#1}{\partial#2}}
\newcommand{\kl}[2]{\mathop{KL}(#1 \parallel #2)}
\newcommand{\nm}{\mathcal{N}}
\newcommand{\sle}{\; \Rightarrow \;}
\newcommand{\indpos}{\mathbf{I}_{d_k}^{+}[i, j]}
\newcommand{\indneg}{\mathbf{I}_{d_k}^{-}[i, j]}

\usepackage{pgfplots}

%my settings
\graphicspath{{../_figures/}}
\usepackage{wrapfig}
\usepackage[font=small]{caption}
\title{Исследование работы линейных моделей}

\author{Тыцкий Владислав}
\date{Ноябрь 2020}

\begin{document}

\maketitle
\section{Градиент в логистической регрессии}
Пусть $X \in R^{NxF}$ -- матрица объектов-признаков, $y \in \{-1,1\}^N$ -- метки соответствующих
объектов, $w \in R^F$ -- вектор весов, $x_i,$ -- i-ый объект $ y_i$ -- метка класса i-ого
объекта,  T - стохастическая подвыборка($|T|$ и T будут обозначаться Т в зависимости от контекста).
Везде будем считать, что добавлен константый признак.

Дана задача оптимизации:
\begin{align}
    \notag & Q(X,y,w) = \mathcal{L}(X,y,w)+\frac{\lambda}{2}||w||_2^2 \rightarrow \min_{w} \\
    \notag & \mathcal{L}(X,y,w) = \frac{1}{T}\sum_{i=1}^{T}\log(1+\exp(-y_i\langle x_i, w\rangle), T \leq N
\end{align}
Для решения этой задачи с помощью градиентых методов необходимо знать градиент функционала $Q(X,y,w)$

\begin{align}
    \notag dQ = &d\mathcal{L} + \frac{\lambda}{2}d\langle w,w \rangle = d\mathcal{L} + \lambda\langle w, dw \rangle \\
    \notag d\mathcal{L} = &\frac{1}{T}\sum_{i=1}^{T}\frac{d(\exp(-y_i\langle x_i, w\rangle))}{1+\exp(-y_i\langle x_i, w\rangle)} =
    \frac{1}{T}\sum_{i=1}^{T}\frac{\exp(-y_i\langle x_i, w\rangle)d\langle-y_ix_i,w\rangle}{1+\exp(-y_i\langle x_i, w\rangle)} = \\
    \notag -&\frac{1}{T}\sum_{i=1}^{T}\frac{\langle y_ix_i,dw\rangle}{1+\exp(y_i\langle x_i, w\rangle)} 
\end{align}
Заметим, что $dQ = \langle \nabla Q,dw\rangle$. Окончательно получаем:
\begin{align}
    \notag \nabla Q(X,y,w) = \lambda w -\frac{1}{T}\sum_{i=1}^{T}\frac{y_ix_i}{1+\exp(y_i\langle x_i, w\rangle)} 
\end{align}
\subsection{Случай для K классов}
Пусть $X \in R^{NxF}$ -- матрица объектов-признаков, $y \in K^N$ -- метки соответствующих
объектов, где $K=\{1\dots k\}$ -- множество классов, $w_i \in R^F$ -- вектор весов соответствующий k-ому классу, 
$x_i$ -- i-ый объект $ y_i$ -- метка класса i-ого
объекта соответственно, T - стохастическая подвыборка($|T|$ и T будут обозначаться 
Т в зависимости от контекста).

Дана задача оптимизации --- максимизация правдоподобия:
\begin{align}
    \notag & Q(X,y,w) = -\frac{1}{T}\sum_{i=1}^{T}\log\mathbb{P}(y_i|x_i)+
    \frac{\lambda}{2}\sum_{k=1}^{K}||w_k||_2^2 \rightarrow \min_{w_1\dots w_k} \\
    \notag & \mathbb{P}(y=c|x) = \frac{\exp{\langle w_c,x \rangle}}{\sum_{k=1}^{K}\exp{\langle w_k,x \rangle}}
\end{align}
Найдем градиент по $w_m$.
\begin{align}
    \notag & dQ(X,y,w) = d(-\frac{1}{T}\sum_{i=1}^{T}\log\mathbb{P}(y_i|x_i)) + d(\frac{\lambda}{2}\sum_{k=1}^{K}||w_k||_2^2) =
    -\frac{1}{T}d(\sum_{i=1}^{T}\log\mathbb{P}(y_i|x_i)) + \lambda(w_m,dw_m) \\
    \notag & d(\sum_{i=1}^{T}\log\mathbb{P}(y_i|x_i))_{w_m} = \sum_{i=1}^{T}d(\log\exp{\langle w_{y_i}, x_i \rangle})-
    \sum_{i=1}^{T}d(\log \sum_{k=1}^{K}\exp{\langle w_k, x_i \rangle}) = \\
    \notag =&\sum_{\substack{i:y_i=w_m \\ i\in T}}\langle x_i,dw_m \rangle - 
    \sum_{i=1}^{T}\frac{\exp{\langle w_m,x_i\rangle}\langle x_i,dw_m \rangle}{\sum_{k=1}^{K}\exp{\langle w_k, x_i \rangle}}
\end{align}
Отсюда получаем: 
$$
\nabla Q_{w_m} =\lambda w_m + \frac{1}{T}\sum_{i=1}^{T}
\frac{\exp{\langle w_m,x_i\rangle}x_i}{\sum_{k=1}^{K}\exp{\langle w_k, x_i \rangle}}-
\frac{1}{T}\sum_{\substack{i:y_i=w_m \\ i\in T}}x_i 
$$
\subsection{Эквивалетность бинарной логистической регрессии и мультиномиальной при K=2}
\noindent \textit{Доказательство}.

\noindent Пусть $w_{+}$ --- вектор весов соответствующий первому классу, а $w_{-}$ --- -1 классу.

\noindent Введем $w = w_{+}-w_{-}$.
\noindent Рассмотрим задачу мультиномиальной регрессии при K=2:

\begin{align}
    \notag & Q(X,y,w) = -\frac{1}{T}\sum_{i=1}^{T}\log
    \frac{\exp{\langle w_c,x \rangle}}{\sum_{k=1}^{K}\exp{\langle w_k,x \rangle}}=
     = - \frac{1}{T}\sum_{\substack{i:y_i=w_{+} \\ i\in T}}
    \log\frac{\exp{\langle w_{+}, x_i \rangle}}{\exp{\langle w_{+}, x_i \rangle}+\exp{\langle w_{-}, x_i \rangle}}\\
    \notag - &\frac{1}{T}\sum_{\substack{i:y_i=w_{-} \\ i\in T}}
    \log\frac{\exp{\langle w_{-}, x_i \rangle}}{\exp{\langle w_{+}, x_i \rangle}+\exp{\langle w_{-}, x_i \rangle}} = 
     - \frac{1}{T}\sum_{\substack{i:y_i=w_{+} \\ i\in T}}\log\frac{1}{1+\exp{\langle -w, x_i \rangle}}- \\
    \notag- &\frac{1}{T}\sum_{\substack{i:y_i=w_{-} \\ i\in T}}\log\frac{1}{\exp{\langle w, x_i \rangle} + 1} = 
    = \frac{1}{T}\sum_{i=1}^{T}\log(1+\exp(-y_i\langle w,x_i \rangle))
\end{align}
То есть функции потерь для бинарной логрегрессии и мультиномиальной регрессии 
при K=2 эквиваленты.

\noindent\textit{ч.т.д}
\section{Задание №1}
\noindent Предобработка документов легко делается с помощью модуля re и метода apply из pandas.
\begin{lstlisting}[language=Python, caption=Clear documents]
    X_train_df.apply(lambda x: re.sub("[^a-zA-z0-9]", " ", x.lower()))
    X_test_df.apply(lambda x: re.sub("[^a-zA-z0-9]", " ", x.lower()))
\end{lstlisting}
\section{Задание №2}
\noindent Использование CountVectorizer для представления слов с помощью bag of words.

\begin{lstlisting}[language=Python, caption=Vectorizer]
vectorizer = CountVectorizer(lowercase=True, min_df=50)
X_train_v = vectorizer.fit_transform(X_train["comment_text"])
X_test_v = vectorizer.transform(X_test["comment_text"])
\end{lstlisting}

\noindent Min\_df = 50 имеет под собой основание.''Оскорбительные'' слова в данном датасете 
встречаются обычно чаще 100 раз. Листинг ниже (\ref{code:count_bad}) демонстрирует код для подсчета.

\begin{lstlisting}[language=Python, caption=Count bad words, label=code:count_bad]
text = X_train["comment_text"]
count = 0
count_bad = 0
for i in range(text.size):
    if text[i].find("very very bad word") != -1:
        count += 1
        if X_train["is_toxic"][i]:
            count_bad += 1
\end{lstlisting}

\section{Задание №3, №4, №5}
Исследуем как ведет себя метод (стохастического) градиентого спуска. 
Я посчитал, что 3, 4, 5 задания можно совместить в одно большое задание --- легче прослеживается
логика повествования.
\footnote{Все графики строились на весьма урезанной по количеству признаков выборке (2300).
 Это сделано для того, чтобы вычислительной мощности компьютера Тыцкого.В.И. хватило построить
 их за разумное время.}

\subsection{Начальная инициализация}
Интересно взглянуть как влияет начальная инициализация весов на функцию потерь.Таблице \ref{pic:weights}
\footnote{Я так и не понял как поменять тип caption с Таблицы на Рисунок}
\newpage
\begin{table}[htb]
    \centering
    \tabcolsep = -10pt
    \begin{tabular}{cc}
        \includegraphics[width=9cm]{/task_3/weights.pdf}  & \includegraphics[width=9cm]{/task_4/weights.pdf} \\
        GD & SGD
    \end{tabular}
    \caption{Зависимость функции потерь от начальной инициализации и итераций}
    \label{pic:weights}
\end{table}

Можно заметить, что внезависисмости от начальной иниципализации спустя небольшое
количество итераций(эпох) функция потерь становится примерно одинаковой.

\textbf{В других экспериментах будем использовать единичный вектор в качестве начальной инициализации}
\subsection{Параметры задающие скорость обучения}
В эспериметнах используется Линейных классификатор, который вычисляет новый вес $w^{i+1}$ 
по формуле:

$$
    w^{i+1} = w^i - \eta  \nabla Q, \:
    \eta = \frac{\alpha}{i^\beta}
$$
где Q - градиент функции потерь.

\noindent Небоходимо понять как влияют гиперпараметры $\alpha$ и $\beta$ на алгоритм.
\noindent В Таблице \ref{pic:GD_alpha_beta} представлены 
графики зависимости функции потерь от параметров $\alpha$ и $\beta$.

\begin{table}[htb]
    \centering
    \tabcolsep = -10pt
    \begin{tabular}{ccc}
        \includegraphics[width=6cm]{/task_3/sub_1/fig_a_0.1_b_1.pdf}  & 
        \includegraphics[width=6cm]{/task_3/sub_1/fig_a_1.0_b_1.pdf} &
         \includegraphics[width=6cm]{/task_3/sub_1/fig_a_10.0_b_1.pdf} \\
         $\alpha = 0.1$ & $\alpha = 1.0$ & $\alpha = 10.0$
    \end{tabular}
    \caption{Зависимость функции потерь от alpha и beta для GD}
    \label{pic:GD_alpha_beta}
\end{table}

\newpage

Заметим как быстро алгоритм отстанавливается, при $\beta>0$. Если и имеет смысл 
использовать ненулевые $\beta$, то только для больших значений параметра $\alpha$.
В то же время при сильно больших $\alpha$ и $\beta=0$ градиентный спуск может не
спуститься в точку экстремума, что плохо сказывается на качестве модели.

\textbf{В дальнейших эспериментах будем брать $\alpha<1$ и $\beta=0$.}

Для стохастического градиентного спуска картина такая же
(Таблица \ref{pic:SGD_alpha_beta}) за исключением того, что при 
больших $\alpha$ поведение еще более непредсказуемо. 

\begin{table}[htb]
    \centering
    \tabcolsep = -10pt
    \begin{tabular}{ccc}
        \includegraphics[width=6cm]{/task_4/sub_1/fig_a_0.1_b_1.pdf}  & 
        \includegraphics[width=6cm]{/task_4/sub_1/fig_a_1.0_b_1.pdf} &
         \includegraphics[width=6cm]{/task_4/sub_1/fig_a_10.0_b_1.pdf} \\
         $\alpha = 0.1$ & $\alpha = 1.0$ & $\alpha = 10.0$
    \end{tabular}
    \caption{Зависимость функции потерь от alpha и beta для SGD}
    \label{pic:SGD_alpha_beta}
\end{table}

\subsection{Сравнение GD и SGD в скорости}
На предущих графиках(Таблица \ref{pic:GD_alpha_beta} Таблица \ref{pic:SGD_alpha_beta})
можно пронаблюдать скорость обучения GD и SGD классификатора. GD делает более ''точныe''
шаги градиентного спуска, но скорость выполнения этого шага довольно медленная. 
Хоть SGD чуть менее точен(совсем незначительно), но из-за того,
что он делает больше шагов градиентного спуска за эпоху, он быстрее сходится к локальному экстремуму. 
Важно заметить, что выбор в пользу SGD сделан конкретно для данного датасета. Для других задач 
поведение GD и SGD может быть совсем разное.

\textbf{В дальнейших экспериментах будем использовать SGD классификатор.}

\subsection{Время на одну итерацию(эпоху)}
Важно оценить время работы классификатора в зависимости от итераций(эпох), чтобы подобрать
оптимальное по соотношению качество скорость итераций(эпох). Таблица \ref{pic:time_iter}.
Одна итерация(эпоха) делается чуть меньше, чем за секунду.

\begin{table}[htb]

    \centering
    \tabcolsep = -10pt
    \begin{tabular}{cc}
        \includegraphics[width=6cm]{/task_3/sub_1/fig_a_0.1_b_1.pdf}  & 
        \includegraphics[width=6cm]{/task_3/sub_2/fig_a_0.1_b_1.pdf} \\
        Time & iterations GD \\
         \includegraphics[width=6cm]{/task_4/sub_1/fig_a_0.1_b_1.pdf} &
         \includegraphics[width=6cm]{/task_4/sub_2/fig_a_0.1_b_1.pdf} \\
         Time & epoch SGD
    \end{tabular}
    \caption{Зависимость между временем и итерацией(эпохой)}
    \label{pic:time_iter}
\end{table}


\subsection{Размер батча для SGD}
\begin{wrapfigure}[13]{r}{0.5\textwidth}
    \includegraphics[width=8cm]{/task_4/fig_batch.pdf} 
    \captionsetup{font={scriptsize,it}}
    \caption{Зависимость функции потерь от размера батча}
    \label{pic:batch}
\end{wrapfigure}

В случае выбора SGD в качесте основного алгоритма важно понять какой размер батча(подвыборки)
оптимален. Важна и скорость работы, и точность шагов градиентного спуска. Справа представлен
график Рис.\ref{pic:batch}.


Можно заметить, что уже при размерах батча 500, 2000 скорость сходимости и
точность шагов градиентого спуска приемлимы --- не возникает скачков, как у размера 100
и скорость гораздо выше, чем для размера 10000 или всей выборки.

В угоду точности сходимости можно немного пожертвовать скоростью, поэтому в \textbf{будущих
экспериментах будет использоваться размер батча 2000.}

\subsection{Качество на обучающей выборке}
Уменьшение функции потерь --- не самое важное для нас. Необходимо взглянуть как меняется мера
качества в идеале на отложенной выборке.\footnote{Не стал делать отложенную выборку и мерил на обучающей} 
Ниже представлены графики Таблица \ref{pic:accuracy} для SGD классификатора(batch\_size=2000,
l2\_coef=0.1)

\begin{table}[htb]
    \centering
    \tabcolsep = -10pt
    \begin{tabular}{ccc}
        \includegraphics[width=6cm]{/task_4/sub_3/fig_a_0.1_b_1.pdf}  & 
        \includegraphics[width=6cm]{/task_4/sub_3/fig_a_1.0_b_1.pdf} &
         \includegraphics[width=6cm]{/task_4/sub_3/fig_a_10.0_b_1.pdf} \\
         $\alpha = 0.1$ & $\alpha = 1.0$ & $\alpha = 10.0$
    \end{tabular}
    \caption{Зависимость функции потерь от alpha и beta для GD}
    \label{pic:accuracy}
\end{table}

Лучшая модель получилась 
\end{document}